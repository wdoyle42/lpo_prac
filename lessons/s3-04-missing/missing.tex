\documentclass[12 pt]{article}
\usepackage{amsmath}
\usepackage{bibentry}
\usepackage{ccaption}
\usepackage{fourier}
\usepackage{stata}
\usepackage[margin= 1in]{geometry}

\usepackage[colorlinks=true,
                      pdfstartview=FitV,
                      urlcolor=blue,
]{hyperref}

\usepackage{natbib}




\title{Dealing with Missing Data Using Multiple Imputation}
\author{William Doyle}
\begin{document}

\thispagestyle{empty}%

\setlength{\parskip}{1ex plus 0.5ex minus 0.2ex}

\setcounter{secnumdepth}{-2}


\begin{flushleft}
  Vanderbilt University\\Leadership, Policy and Organizations\\Class Number 9553\\ Spring 2019\\
\end{flushleft}

\begin{center}
\textbf{Practicum: Dealing with Missing Data Using Multiple Imputation}
\end{center}

\section{The Missing Data Problem}
\label{sec:misss-data-probl}

``The problem with missing data is that it's missing''--Nathaniel Beck

When dealing with a dataset with severe missing data, we are living in
the land of bad options. We want all of our approaches to guarantee
first that our estimates are unbiased, and second, that we can use as
much of the actual data as possible. 

All of the techniques described below are evaluated based first on
whether they introduce bias into the estimates and second on whether
they provide more efficiency.  The goal of techniques for missing data
is not to literally replace the data that isn't there, but to come up
with the least bad way to use that data that is there to estimate the
parameters of interest.

There are three types of missing data:
\begin{enumerate}
\item \textbf{Missing Completely at Random (MCAR)}: In this case, the
  data are missing in a pattern that is independent of other variables
  in the dataset. For instance, a question on a survey might have only
  been asked of a random subset of respondents. Let $Z$ be a variable
  with missing data, and $X$ be a vector of always observed
  variables. If $R_z$ is an indicator variable which 1 in if $Z$ is
  missing and 0 otherwise, then the MCAR assumption can be formalized
  as:

  \begin{equation*}
    Pr(R_z=1|X,Z)=Pr(R_Z=1)
  \end{equation*}


\item \textbf{Missing at Random (MAR)} In this case, the data are
  missing in such a way that they are dependent on at least some of
  the other available data, including the data missing from the
  variable itself. Importantly, the missing data are dependent on the
  \textit{observed} values of the other data. The relationship between
  the missing data and the available data is not deterministic. Using
  the same notation as before, the MAR assumption can be formalized
  as:

  \begin{equation*}
    Pr(R_z=1|X,Z)=Pr(R_z=1|X)
  \end{equation*}

It's important to note that the missing information in $Z $can depend
on$X$, but not on $Z$ alone, after you have conditioned on $X$. This
actually shares some characteristics with the conditional independence
assumption used in causal inference. 

The ``at random'' part of missing at random does not mean
\textit{completely} random. It means that the indicator variable for
the missing data is a random variable whose conditional mean is
governed by other data in the dataset. 

\item \textbf{Not Missing at Random (NMAR) } In this case, the data are
  missing as a function of the underlying variable which is missing data. For instance, if
  respondents consistently refused to report their own income above a certain
  income level. Data not missing at random are dependent on the values
  of the unobserved data. This is also called ``non-ignorable''
  missingness. 

Formally, if we are missing data for a variable $Z$ for all $Z>A$,

\begin{equation*}
  Pr(R_z=1|Z>A)=1
\end{equation*}

\end{enumerate}


If we have MCAR data, it's just a subsample of the sample, and there's
no really problem. If we have NMAR data, there's nothing to be done. 

Most of the time, we're dealing with data that's MAR---we know it's
not missing completely at random, but we're not stuck with data that
has a perfect relationship with values of the underlying variable, and
can be reasonably predicted from other observed data. 

The following is a list of commonly used options for missing data and
reasons not to use any of them except multiple imputation (and its
variants). 

\section{Casewise Deletion (Complete-Case Analysis)}
\label{sec:casew-delet-compl}

You should be familiar with casewise deletion at this point. This is
the default option in Stata and other similar statistical programs. Casewise
deletion is acceptable only when the data are MCAR. Essentially, the
analyst is completing the analysis on a random sample of a random
sample, which is not problematic.  Other names for this are listwise
deletion or complete case analysis. 

When do we know this is true? Generally, only when the sample design
results in missing data patterns. For instance, if a portion of a
questionnaire was applied only to a random subset of the respondents,
then casewise deletion based on missing data is appropriate.

In any circumstance where the missing data were not generated
intentionally, you can not safely assume MCAR, and therefore can not
use casewise deletion. In any case where MCAR is not exactly
established, casewise deletion results in both bias and a loss of
efficiency. Bias results because the estimates are derived from a
non-random sample, loss of efficiency because casewise deletion by
design throws out a large amount of data. You should only use casewise
deletion in cases when a very small amount of data are missing, and
you are quite uncertain about how to model the missing data (these
circumstances are vanishingly rare in practice). 

However, even though this is an \emph{inefficient} method, it does not
appear to bias results much. Allison writes:

\begin{quote}
  Somewhat surprisingly, listwise deletion is very robust to
  violations of MCAR (or even MAR) for predictor variables in a
  regression analysis. Specifically, so long as missingnenss on the
  predictors does not depend on the dependent variable, listwise
  delation will yield approimately unbiased estimates of regression
  coefficients (Little, 1992). And this holds for virtually any kind
  of regression--linear, logistic, Poisson, Cox, etc. (Allison 2008
  p. 75)
\end{quote}

\section{Weighted Casewise Deletion}
\label{sec:weight-casew-delat}

Weighted casewise deletion attempts to make up for the loss of data
from casewise deletion by re-weighting the sample to account for the
patterns of missing data. Each complete case is assigned a new weight
after deletion is complete, resulting in a weighted sample that is
supposed to represent the overall population under study. 

The same conditions must apply for weighted casewise deletion to be
appropriate as must apply when casewise deletion is
appropriate. Otherwise, the bias caused will be even worse, since the
technique assumes a subset of a random sample as the basis for
re-weighting. 

Surprisingly, this is \emph{exactly} what NCES does with many of its
longitudinal datasets. The ``panel weights'' applied to various
combinations of panels simply re-weight the data in order to reflect
the new pattern of missing data. 


\section{Mean Imputation}
\label{sec:mean-imputation}

Mean imputation is never a good idea. Also called ``mean-plugging'',
mean imputation involves replacing missing data for the variable
$x_{i1}$ with the mean of that variable, $\bar{x_1}$. As you can
probably sense intuitively, this drastically reduces that estimates of
the variance of $x_1$, since we're artificially constraining the
variance. Thus, all variance estimates will be attenuated. As Little
and Rubin say ``This method cannot be recommended''. (p. 62)


\section{Mean Imputation with a Dummy}
\label{sec:mean-imputation-with}

There are multiple proposed methods for mean plugging that involve
including a dummy variable for whether or not the individual variable
has been mean-plugged. The idea is to condition on imputation, to make
up for the fact that we've imputed that particular value. Conditioning
on imputation does not help with the bias associated with mean
plugging, although just by virtue of variance inflation it does reduce
the problem of overconfidence slightly. This method is also not
recommended under any circumstances. 


\section{Conditional Mean Imputation}
\label{sec:cond-mean-imput}

Conditional mean imputation involves taking a model for the missing
data, which we can state in general form as $E (x_1 |x_2, x_3, \ldots x_p)$.
This is a little better than unconditional mean imputation, but not
much. Essentially this procedure ignores the standard error of the
regression, which means that predictions are based only on the
systematic relationship between $x_1$ and the other variables. You can
add in an error term --this is called stochastic regression
imputation. Also called Buck's method, this is a ``not-too-bad'' approach.
However, we can do better than this.

\section{Hotdecking, cold decking}
\label{sec:hotdecking}

In hot decking, missing data are replaced by values from other,
similar units in the sample. The term comes from the days of computer
cards, when the analyst would grab a card from another unit from the
``hot deck'' and put it back into the cold deck for reanalysis. As you
might imagine, the key to hotdecking is deciding which units are
``similar'' to the unit that's missing data. The properties of these
methods are not well established, but theoretical and empirical
investigations indicate that estimates from data which have been
imputed using hotdeck methods are unbiased only when the MCAR
assumption holds. Hot decking is also quite commonly used in NCES
adminstrative surveys. 

Cold-decking means taking data from an outside data set (usually a
previous survey) and carrying that data into the current

\section{Logical Imputation}
\label{sec:logical-imputation}

In certain limited situations we may logically infer the value of a
missing response based on patterns of other responses. For instance,
if an individual has indicated that they are 31, we may logically
infer that they do not receive Medicare part D benefits.  


\section{Multiple Imputation}
\label{sec:multiple-imputation}
After their review of the above methods, Little and Rubin conclude: 

\begin{quote}
  Imputations should generally be:
  \begin{itemize}
  \item Conditioned on observed variables, to reduce bias due to
    nonresponse, improve precision, and preserve association between
    missing and observed variables;


  \item Multivariate, to preserve associations between missing
    variables;


  \item Draws from the predictive distribution rather than means, to
    provide valid estimates of a wide range of estimands. (Little and
    Rubin (2002) p. 72) 
  \end{itemize}
\end{quote}

As it turns out, multiple imputation provides the only method that
meets all of these goals. 

We'll talk about two algorithms for multiple imputation: Multiple
imputation via chained equations (mice) and data augmentation (DA). 

\subsection{Multiple Imputation via Chained Equations (MICE)}
\label{sec:mult-imput-via}

In multiple imputation via chained equations, we use separate models
for each missing variable, then use the newly imputed data for that
variable to influence the estimates for the next variable. 


Say we had $p$ variables in our dataset, each of which was missing some data. First,
we would choose values for the missing data $x_1^0 \ldots x_p^0$,
usually at random. Then we would iterate across several cycles denoted
by $t$, with new values drawn from the conditional distribution for
each variable:

\begin{align*}
  x_1^{(t+1)}&=p(x_1|x_2^{(t)},x_3^{(t)},\ldots x_p^{(t)})\\
  x_2^{(t+1)}&=p(x_2|x_1^{(t+1)},x_3^{(t)},\ldots x_p^{(t)})\\
  x_3^{(t+1)}&=p(x_3|x_1^{(t+1)},x_2^{(t+1)},\ldots x_p^{(t)})\\
  x_p^{(t+1)}&=p(x_p|x_1^{(t+1)},x_2^{(t+1)},\ldots x_{(p-1)}^{(t)})
\end{align*}


The steps for MICE are as follows:

\begin{enumerate}
\item Create a model for each missing variable.


\item Run an algorithm that creates estimates for each of the models,
  which will iterate across all of the variables.


\item Take a random draw for the missing variable from the predictions
  from the above models. 


\item Take the draw and use it to fill in the missing data point--this
  is called imputation based on the predictive distribution.


\item Repeat the above steps. Theoretical and empirical results
  indicate that between 2-10 repetitions are appropriate, depending on
  the complexity of the models and the amount of missing data. This
  process is what is known as multiple imputation. In typical usage we
  impute 5 different datasets. 


\item Run estimates an each of the separate imputed datasets. These
  are then combined. 
\end{enumerate}

\subsection{Multiple Imputation via Data Augmentation}
\label{sec:mutl-imput-via}


The DA algorithm is used by most software for multiple imputation. It
works like this: we fist come up with a starting point for the
coefficients for the ``main'' regression. We then use the variance
covariance matrix from these parameters to obtain coefficients from
regressions in which the dependent variable is the variable missing
data-- going from the first to the last variable missing any data. We
then come up with a prediction for a given variable based on the other
data. To this prediction, some random noise is added.  This new data
is added is added to the completed data set to come up with a new
variance covariance matrix. From this  This is iterated, until the models converge. 

\subsection{Imputation}

Based on the results of either of the above procedures, we can take a draw from the
conditional distribution of each $x_p$ and plug it into the
dataset. For each missing data point, we plug in a different draw,
resulting in multiple sets of data. 


The steps for multiple imputation via data augmentation are as
follows (Allison, 2002):

\begin{enumerate}
\item Select start values for your parameters. Stata does this using
  the EM algorithm. 

\item Use the current values of the means and covariances to obtain
  estimates of regression coefficients for equations in which each
  variable with missing data is regressed on all observed variables. 

\item Use the regression estimates to generate predicted values for
  all missing values. To each predicted value, add a random draw based
  on the distribution of that variable.

\item With the newly completed dataset, recalculate the
  variance-covariance matrix. 

\item With the new variance-covariance matrix, take a random draw from
  the possible values for the means and covariances.

\item Iterate, starting at step 2. 
\end{enumerate}



\subsection{Combining Results from Imputed Datasets}

The analysis is conducted as normal for each imputed dataset. The
question is then how to combine the results? Combining parameter
estimates is easy: it's the mean of the estimates from all of the
datasets.

Variance estimates are a little harder to combine, although it's
really not too bad. The key is to reflect the within-imputation
variance from each run of the model with the between imputation
variance across all models. Take $D$ to be the count of imputed
datasets, and $V_d$ is the variance estimate from dataset $d$. If $\theta$ is the estimate we're
interested in, and $\hat{\theta}_d$ is our estimate from dataset $d$, Rubin's
variance estimate is: 

\begin{equation*}
  Var(\theta|X_{obs})\approx
  \frac{1}{D}\sum_{i=1}^{D}V_d+\frac{1}{D-1}\sum_{i=1}^{D}(\hat{\theta}_d-\bar{\hat{\theta}})^2= \bar{V}+B
\end{equation*}

Where $\bar{V}$  is the average of all variance estimates and $B$ is
the between-imputation variance. 

Luckily, we usually don't have to do this by hand, but it's good to
know how to do so, since many times ``canned'' missing data programs
might get stuck on a particular model we're using.

\section{Using Stata's missing data commands}

Stata has a very strong set of imputation commands. The first thing
you'll want to do is to run some basic descriptives to understand the
extent of your missing data problem. The most frequent question I get
is: how much is too much missing data? There's no one answer to this
question. If you have more than 20\% of observations missing for more
than 50\% of your analysis variables, you have a fairly serious
problem. You can run all of the missing data routines described here,
but it just won't help, because you don't have enough data. The next
thing to do is to run the \texttt{mvpatterns} command. The output of
\texttt{mvpatterns} looks like this:

\begin{stlog}
  
Patterns of missing values

  +---------------------------------------------------------------------+
  |                                              _pattern   _mv   _freq |
  |---------------------------------------------------------------------|
  | +++++++++++++++++++++++++++++++++++++++++++++++++++++     0   11638 |
  | .++++++++++++++++++++++++++++++++++++++++++++++++++++     1    1727 |
  | +.........++.........................................    50     477 |
  | +....................................................    52     171 |
  | ..........++.........................................    51     117 |
  |---------------------------------------------------------------------|
  | +++.+++++++++++++++++++++++++++++++++++++++++++++++++     1      39 |
  | ++..+++.............++++++++++++++++++++++++........+    23      31 |
  | ++..................................................+    50      24 |
  | ++..++..............++++++++........++++++++........+    32      11 |
  | .+..++..............++++++++........++++++++........+    33       6 |
  |---------------------------------------------------------------------|
  | .+..................................................+    51       5 |
  | .+..+++.............++++++++++++++++++++++++........+    24       4 |
  | .++.+++++++++++++++++++++++++++++++++++++++++++++++++     2       3 |
  | ++..+.+.....................++++++++++++++++........+    32       3 |
  +---------------------------------------------------------------------+

\end{stlog}

This tells you how missing data patterns may be related across
variables. 

With that, you're set to begin running the \texttt{mi} set of
commands. The first thing to do is to tell Stata that you're going to
create a ``long'' dataset, stacking each imputation, one under the
other:

\texttt{mi set mlong}

You next to ``register'' the data, telling Stata which variables with
missing data are going to predicted by which variables without missing
data. 

With that, you're ready to run the \texttt{mi impute} command. The
\texttt{mi impute} command has a variety of ways of using the existing
data to predict the missing data. We're going to use \texttt{mi mvn}
which makes a broad assumption of multivariate normaility in order to
use some advanced MCMC techniques. 


\subsection{Using mi impute chained}
\label{sec:using-mi-impute}

The \texttt{mi impute chained} command works by specifying a model for
each of the missing variables that you want to be imputed. It then
runs an iterated series of predictions of each of the missing
variables.  You'll need to specify several options.

\begin{description}

\item The models to be used for different groups of variables
\item  The number of imputations to be added \texttt{add}. Use 5 as
  the default. 
\item The number of ``burn in'' iterations to go through. 100 is
  recommended. 
\item Customizations of the default prediction equations.
\end{description}


\subsection{Using mi impute mvn}

The \texttt{mi impute mvn} command works in two steps. In the first
step, it uses the expectation-maximization algorithm to find some
reasonable starting points. In the second step, it uses a Data
Augmentation routine to fill in the missing data in the iterative
manner described above. 

There are several options that you need to specify before using
\texttt{mi mvn}:

\begin{description}
  \item The number of imputations to be added, add(). 


    \item The ``prior'' to be used. The prior provides a probabilisitc
      description of where you expect the missing data points to be. I
      recommend a uniform prior. 


    \item The number of ``burn in'' iterations that the DA algorithm
      should run before you start using the results. Something like
      1,000 should work well. 


    \item The number of ``burn between'' iterations that should be
      used between each dataset included in the multiply imputed
      results. 


      \item How the mcmc chains should be initialized: use the em
        algorithm. 


        \item You can also use save some information about the
          convergence of the mcmc run. 
\end{description}

Once you have your multiply imputed dataset ready to go, you can then
use the \texttt{mi estimate :} prefix before commands.  This will use
all of your imputed datasets in estimation, and return results that
are averaged across all of the imputed datasets. In the do file, note
that I have included a code ``chunk'' that allows the results from mi
estimate and svy to be outputted to the estout command. 

In the output from \texttt{mi impute mvn} you'll first see the results
of the em algorithm:

\begin{stlog}
  . mi impute mvn /*Assuming mvn pattern, can use multiple methods to predict*/
> `race' `pared' byses1
> =
> f1psepln byincome 
> 
> /*Nonmissing data*/
> 
>   
> 
> ,  add(5)  /*Number of imputations: Use 5 to start*/
> alldots
> noisily
> prior(ridge, df(0.5))
> burnin(2000)
> burnbetween(500)
> initmcmc(em,  iter(2000) tol(1e-6))
> savewlf(wlf, replace)
> force
> ;

Performing EM optimization:
note: 849 observations omitted from EM estimation because of all imputation variables missing

Iteration 0:   Observed log posterior =  68051.922
Iteration 1:   Observed log posterior =  69071.266

Expectation-maximization estimation      Number obs           =     13407
                                         Number missing       =         0
                                         Number patterns      =         1
Prior: ridge, df=.5                      Obs per pattern: min =     13407
                                                          avg =     13407
                                                          max =     13407

Observed log posterior =  69071.266 at iteration 1

--------------------------------------------------------------------------------
             |     amind      asian      black   hispanic  multira~l     byses1 
-------------+------------------------------------------------------------------
Coef         |                                                                  
    f1psepln | -.0026674    .024547   .0157179  -.0159283  -.0031649   .1104196 
    byincome | -.0014871   -.010512  -.0289848  -.0222282   .0005692   .2116825 
       _cons |  .0338002   .0844891   .3222665   .4119973   .0540025  -2.350626 
-------------+------------------------------------------------------------------
Sigma        |                                                                  
       amind |  .0081838  -.0008288  -.0012732  -.0013977   -.000378  -.0001191 
       asian | -.0008288    .086905  -.0140171  -.0141417  -.0043064   .0024234 
       black | -.0012732  -.0140171   .1057976  -.0207833   -.005605  -.0010089 
    hispanic | -.0013977  -.0141417  -.0207833   .1149542  -.0061889  -.0267954 
 multiracial |  -.000378  -.0043064   -.005605  -.0061889   .0430571   .0011872 
      byses1 | -.0001191   .0024234  -.0010089  -.0267954   .0011872   .2611111 
--------------------------------------------------------------------------------

\end{stlog}


Next, Stata runs through the ``burn in'' period for the mcmc
algorithm:

\begin{stlog}
  Performing MCMC data augmentation: 
  burn-in 2000 .........10.........20.........30.........40.........50.........60.........70.........80.........90.........100...
> ......110.........120.........130.........140.........150.........160.........170.........180.........190.........200.........2
> 10.........220.........230.........240.........250.........260.........270.........280.........290.........300.........310.....
> ....320.........330.........340.........350.........360.........370.........380.........390.........400.........410.........420
> .........430.........440.........450.........460.........470.........480.........490.........500.........510.........520.......
> ..530.........540.........550.........560.........570.........580.........590.........600.........610.........620.........630..
> .......640.........650.........660.........670.........680.........690.........700.........710.........720.........730.........
> 740.........750.........760.........770.........780.........790.........800.........810.........820.........830.........840....
> .....850.........860.........870.........880.........890.........900.........910.........920.........930.........940.........95
> 0.........960.........970.........980.........990.........1000.........1010.........1020.........1030.........1040.........1050
> .........1060.........1070.........1080.........1090.........1100.........1110.........1120.........1130.........1140.........1
> 150.........1160.........1170.........1180.........1190.........1200.........1210.........1220.........1230.........1240.......
> ..1250.........1260.........1270.........1280.........1290.........1300.........1310.........1320.........1330.........1340....
> .....1350.........1360.........1370.........1380.........1390.........1400.........1410.........1420.........1430.........1440.
> ........1450.........1460.........1470.........1480.........1490.........1500.........1510.........1520.........1530.........15
> 40.........1550.........1560.........1570.........1580.........1590.........1600.........1610.........1620.........1630........
> .1640.........1650.........1660.........1670.........1680.........1690.........1700.........1710.........1720.........1730.....
> ....1740.........1750.........1760.........1770.........1780.........1790.........1800.........1810.........1820.........1830..
> .......1840.........1850.........1860.........1870.........1880.........1890.........1900.........1910.........1920.........193
> 0.........1940.........1950.........1960.........1970.........1980.........1990.........2000 done
\end{stlog}

Last, it imputes the data, iterating the number of times specified in
the option burnbetween:

\begin{stlog}
    imputing m=1 through m=5 ..... done
\end{stlog}

Last, it gives you a summary of the imputation procedure:

\begin{stlog}
  Multivariate imputation                 Imputations =        5
Multivariate normal regression                added =        5
Imputed: m=1 through m=5                    updated =        0

Prior: ridge, df=.5                      Iterations =     4000
                                            burn-in =     2000
                                            between =      500

               |              Observations per m              
               |----------------------------------------------
      Variable |   complete   incomplete   imputed |     total
---------------+-----------------------------------+----------
         amind |      13407          849       849 |     14256
         asian |      13407          849       849 |     14256
         black |      13407          849       849 |     14256
      hispanic |      13407          849       849 |     14256
   multiracial |      13407          849       849 |     14256
        byses1 |      13407          849       849 |     14256
--------------------------------------------------------------
(complete + incomplete = total; imputed is the minimum across m
 of the number of filled in observations.)

\end{stlog}

\section{A Final Warning}

Multiple imputation is always the LAST STEP in an analysis. Your
design for analysis should be in place, with all models set up and
ready to go. In general, it is far too time consuming to deal with
multiple imputation when you're conducting preliminary analyses. 

Remember, multiple imputation DOES NOT give you a value for any given
missing data point. You should never report individual data points
from a multiply imputed dataset as actual data. Instead, multiple
imputation allows you to estimate results that reflect the actual,
observed data without either losing information or biasing your
results. 

\end{document}

%%% Local Variables: 
%%% mode: latex
%%% TeX-master: t
%%% End: 
