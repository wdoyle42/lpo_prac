\documentclass[12pt]{article}
\usepackage{bibentry}
\usepackage{fourier}
\usepackage[margin=1in]{geometry}

\usepackage[colorlinks=true,
                      pdfstartview=FitV,
                      urlcolor=blue,
]{hyperref}
\usepackage{natbib}



\begin{document}
\bibliographystyle{apalike}
\nobibliography{nature}

\thispagestyle{empty}%


\setlength{\parskip}{1ex plus 0.5ex minus 0.2ex}

\setcounter{secnumdepth}{-2}



\begin{flushleft}
Vanderbilt University\\Leadership, Policy and Organizations\\Class Number 9951\\ Fall 2016\\
\end{flushleft}

\begin{center}
\Large{\textbf{Methods Practicum}}\\
\end{center}

\begin{flushleft}
William R. Doyle\\
Office: 207D Payne\\
Office Hours: Monday and Wednesday 1-3 or by Appointment \\
w.doyle@vanderbilt.edu\\
phone (615) 322-2904\\

\vskip 12 pt

\end{flushleft}

\section{Course Overview}%

The overview includes an introduction to the course, guidelines on grading, and required texts.

\subsection{Introduction}%
\begin{flushleft}

  This course is the first of a three semester series of courses
  designed to introduce you to the \textit{practice} of research,
  particularly the applied side of quantitative research. The goal of
  this course to help you to prepare a paper that can be presented at
  a major research conference and, hopefully, submitted to a journal
  for publication.

  To accomplish this goal, you will choose from among publicly
  available datasets. You will identify a research topic, then later a
  research question. You will create a dataset using the publicly
  available sources.  By the end of the semester, you will have a
  properly formatted and cleaned dataset, with auxiliary information
  from other sources included. Next semester we will analyze this
  dataset. By the end of May, you will complete a paper based on this
  analysis.

  Along the way, you will develop skills that will be helpful in
  future work using any kind of data. This class has a strong emphasis
  on using programming skills to aid in the replication of work and to
  simplify complex analyses.


\end{flushleft}

\subsection{Grading}%
\begin{flushleft}
Evaluation for the course will be based on the following factors:

\textit{Assignments: 50\%}

There will be weekly assignments,
pass/fail. Late assignments will not be accepted. These assignments
will account for half of your grade. Collaboration on assignments is
fine, however, many of the assignments will ask you to work with
variables and datasets of your own choosing.

\textit{Summary Paper and Codebook: 50\%}

At the end of the semester, you will need to present the results of
your data collection efforts, with a summary paper of no more than
five pages of text, accompanied by properly formatted tables and
graphics. The summary paper and codebook will be due on December
4th. You will present your work to the rest of the class on December



\end{flushleft}

\subsection{Texts}%

\begin{flushleft}

The following texts are available in the bookstore:

Baum, C (2006) \textit{An Introduction to Modern Econometrics Using
STATA}. College Station: STATA Press

Long, J.S. (2009) \textit{The Workflow of Data Analysis Using Stata}. College Station: STATA Press

For this semester, both of these books are \textit{optional}: I will
recommend a few chapters from both as the semester progresses, but the
class notes are the only required reading. 

\end{flushleft}

\subsection{Software}

You need to have access to a working version of Stata, (at least v 13.0).
Stata is installed on computers on Peabody campus, including our
classroom (Wyatt 132), and on stations in the Peabody library. You
are not required to purchase Stata, but you will need to use it for
class assignments.

If you do purchase Stata, you will need Stata IC (standard
version). Vanderbilt has what's called a gradplan with Stata under
which you can purchase the software at greatly reduced prices.  Stata
SE is a more-powerful version of Stata that is useful for the larger
datasets many of you may be working with.


\subsection{Honor Code}
\label{sec:honor-code}

For this course, you are bound by the terms of the Peabody Honor
System. Any breach of academic honesty, including cheating,
plagiarism, or failing to report a known or suspected violation of the
Code will be reported to the Honor Council. In particular, papers must
assign credit to the sources you use. Material borrowed from
another--quotations, paraphrases, key words, or ideas--must be
credited following appropriate citation procedures (footnotes and
bibliography). As mentioned above, collaboration \textit{is} permitted
on assignments but \textit{is not} permitted on your summary paper and
codebook. 
 
If you have any doubts, please ask me for clarification. Uncertainty
about the application of the Honor Code does not excuse a violation.

\section{Schedule for Meetings}

The schedule for all class meetings is as follows:

\begin{flushleft}

\subsection{August 21}

\textit{Topics}:

Class introduction

\subsection{August 28}



\textit{Topics}:
STATA Basics

\textit {Assignments:}
Assignment 1 due August 25, midnight

\subsection{Septermber 4}


\textit{Topics}:

Working with NCES databases

\textit {Assignments:}
Assignment 2 due September 1, midnight

\subsection{September 11}


\textit{Topics}:

Dataset manipulation: Collapsing, merging, bending the data to your will


\textit {Assignments:}
Assignment 3 due September 8, midnight
\subsection{September 18}


\textit{Topics}:
More dataset manipulation: One to many merging, many to one merging, appending datasets

\textit {Assignments:}
Assignment 4 due September 15, midnight

Summary research area due

\subsection{September 25}


\textit{Topics}:

Sampling: Simple sampling designs

\textit {Assignments:}
Assignment 5 due September 22, midnight
\subsection{October 2}


\textit{Topics}:
Sampling: Complex sampling designs


\textit {Assignments:}
Assignment 6 due September 29, midnight
\subsection{October 9}


\textit{Topics}:

Data Cleaning

\textit {Assignments:}
Assignment 7 due October 6, midnight

\subsection{October 16}


\textit{Topics}:

Data Validation

\textit {Assignments:}
Assignment 8 due October 13, midnight

Research questions due. 

\subsection{October 23}


\textit{Topics}:

Descriptive Statistics: Tabular and Graphical Approaches

\textit {Assignments:}

Assignment 9 due  October 20, midnight

\subsection{October 30}


\textit{Topics}:

An Introduction to Programming: Using Macros

Baum, C (2005) A little bit of Stata programming goes a long
way. Boston College Working Papers in Economics, 612. pp 1-10 \href{http://fmwww.bc.edu/EC-P/WP612.pdf}{Online} 

\textit {Assignments:}

Assignment 10 due October 27, midnight

\subsection{November 6}


\textit{Topics}:

Lab Hours: We will go over everyone's projects in detail. Come
prepared to discuss your research questions, code, and results of
descriptive analysis.

\subsection{November 13}

\textit{Topics}:

Graphical approaches to describing data 


Gelman, A \&  Pasarica, C \& and Dodhia, Rahul (2002) Let's practice
what we preach: Turning tables into graphs. \textit{American
  Statistician} 56:2 p. 121

\subsection{November 20}


\textit{Topics}:

Topic TBA: based on student work. 


\subsection{November 27}

No Class- Thanksgiving Break

\subsection{December 4}

\textit{Topics}:

Class Presentations

\textit {Assignments:}

Summary Papers, Do-File and Codebooks due before class, December 4. 


\end{flushleft}
\end{document}




















